% \documentclass[oneside,10pt]{article}
% \usepackage[paperheight=118.8mm,paperwidth=68.2mm,margin=2mm]{geometry}
% https://tex.stackexchange.com/questions/108149/how-to-set-size-of-pdf-page-in-pixels-and-put-a-background-image
% \pagestyle{empty}
\documentclass[oneside,12pt]{article}
% \documentclass[oneside]{standalone}
% \usepackage[paperwidth=216.0pt,paperheight=384.0pt,margin=0pt]{geometry}
% \usepackage[paperwidth=1080px,paperheight=1920px,margin=0px]{geometry}
\usepackage[paperwidth=216.0pt,paperheight=384.0pt,margin=5pt]{geometry}
\usepackage[utf8]{inputenc}
\usepackage[english,russian]{babel}
\renewcommand{\familydefault}{\sfdefault}\normalfont
\parindent=0pt

\usepackage{xcolor}
\definecolor{bg}{RGB}{22,22,22}
\definecolor{lightgreen}{RGB}{144,238,144}
\pagecolor{bg}

\usepackage{enumitem}
\setlist[description]{format=\textcolor{yellow!30}}

% \setlength{\oddsidemargin}{0px}
% \usepackage{indentfirst}
\begin{document}
% \textcolor(lightgreen)
% \color(lightgreen)
\setcounter{section}{0}
\section{Носители заряда}

\color{lightgreen}
\begin{description}
    \item[металл и вакуум]\ \\ электроны
    \item[электролит]\ \\ положительные и отрицательные\\ ионы 
    \item[газ/плазма]\ \\ положительные ионы и электроны
    \item[полупроводник]\ \\ электроны и дырки\\ (от электронов)
\end{description}
\end{document}
